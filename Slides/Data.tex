%\setbeameroption{notes on second screen} %Dual-Screen Notes
%\setbeameroption{show only notes} %Notes Output

% white on black
\setbeamercolor{normal text}{fg=white,bg=blue!20!black}
\setbeamercolor{structure}{fg=white}
\setbeamercolor{alerted text}{fg=red!85!black}
%\setbeamercolor{item projected}{use=item,fg=black,bg=item.fg!35}
\setbeamercolor*{palette primary}{use=structure,fg=structure.fg}
\setbeamercolor*{palette secondary}{use=structure,fg=structure.fg!95!black}
\setbeamercolor*{palette tertiary}{use=structure,fg=structure.fg!90!black}
\setbeamercolor*{palette quaternary}{use=structure,fg=structure.fg!95!black,bg=black!80}
\setbeamercolor*{framesubtitle}{fg=white}
\setbeamercolor*{block title}{parent=structure,bg=black!60}
\setbeamercolor*{block body}{fg=black,bg=black}
\setbeamercolor*{block title alerted}{parent=alerted text,bg=black!15}
\setbeamercolor*{block title example}{parent=example text,bg=black!15}

% black on white
%\usetheme{Singapore} %Gray with fade at top
%\usecolortheme{seagull} %Color theme
%\usecolortheme{rose} %Inner color theme
\setbeamercolor{item}{fg=black!65}
%\setbeamercolor{enumerate item}{fg=black!55}

\useoutertheme[subsection=false]{miniframes} %Supppress subsection in header
\useinnertheme{rectangles} %Itemize/Enumerate boxes
\setbeamertemplate{navigation symbols}{}
\setbeamertemplate{mini frames}[default]
\setbeamercovered{dynamics}
\setbeamerfont*{title}{size=\Large,series=\bfseries}
%\setbeamertemplate{headline}{}

\newcommand{\heading}[1]{\noindent \textbf{#1}\\ \vspace{1em}}

\usepackage{bbding,color,multirow,times,ccaption,tabularx,graphicx,verbatim} %graphics,
\usepackage{colortbl}%Table overlays
\usepackage[english]{babel}

% font
\usepackage{tgadventor}
\renewcommand*\familydefault{\sfdefault} %% Only if the base font of the document is to be sans serif
\usepackage[T1]{fontenc}


\title[]{Loading and Manipulating Data}

\begin{document}

\begin{frame}
	\titlepage
\end{frame}

\frame{}
\frame{\tableofcontents}

\section{Loading Data}
\frame{\tableofcontents[currentsection]}


\frame{
\frametitle{Loading Data}
\begin{itemize}\itemsep2em
\item Typically use data in a ``dataframe''
\item Not restricted to one dataframe at a time
\item Functions to load data all create a dataframe
\item Statistical functions accept vectors or dataframes
\end{itemize}
}

\frame{\frametitle{Try on your own}
Understand dataframe objects:
\begin{itemize}\itemsep1em
\item \href{https://github.com/leeper/Rcourse/blob/gh-pages/Scripts/dataframe-structure.r}{``Dataframe Structure''}
\item \href{https://github.com/leeper/Rcourse/tree/gh-pages/Scripts/dataframe-arrangement.r}{``Rearranging Dataframes''}
\end{itemize}
}


\frame[label=q]{\Large Questions so far?}


\frame{
\frametitle{Loading Data}
\begin{itemize}\itemsep2em
\item There's no ``open'' button
\item A functions for each file format:
	\begin{itemize}
	\item CSV: \texttt{read.csv}
	\item TSV: \texttt{read.delim}
	\item Stata: \texttt{read.dta} (from \textbf{foreign})
	\item SPSS: \texttt{read.spss} (from \textbf{foreign})
	\end{itemize}
\item Almost anything can be loaded
\end{itemize}
}

\frame{\frametitle{Try on your own}
Understand dataframe objects:\\
Do the \href{https://github.com/leeper/Rcourse/blob/gh-pages/Scripts/loadingdata.r}{``Loading Data''} Tutorial
}

\againframe{q}



\section{Basic Data Summaries}


\frame{
\frametitle{Summary Statistics}
\begin{itemize}\itemsep2em
\item Lots of built-in functions to summarize data
\item One important function: \texttt{summary}
\end{itemize}
}


\frame{\frametitle{Try on your own}
Understand basic data summaries:
\begin{itemize}\itemsep1em
\item \href{https://github.com/leeper/Rcourse/blob/gh-pages/Scripts/univariate.r}{``Univariate data summaries''}
\item \href{https://github.com/leeper/Rcourse/tree/gh-pages/Scripts/correlation.r}{``Correlations''}
\end{itemize}
}

\frame{
\frametitle{Summary Tables}
\begin{itemize}\itemsep2em
\item Tabulation is easy with \texttt{table}
\item Creates univariate tables and cross-tables
\item Tables are objects (of class ``table''), so we can work with them like any other object
\end{itemize}
}


\frame{\frametitle{Try on your own}
Understand tabulation and cross-tabulation:
\begin{itemize}\itemsep1em
\item \href{https://github.com/leeper/Rcourse/tree/gh-pages/Scripts/tables.r}{``Tabulation''}
\end{itemize}
}




\frame{
\frametitle{Summary Plots}
\begin{itemize}\itemsep2em
\item Visualization is one of R's greatest strengths
\item One important function: \texttt{plot}
\item Many other functions for specific types of plots
\item Basic plots look okay
\item Plots can be made beautiful with a little work
\end{itemize}
}


\frame{\frametitle{Try on your own}
Understand basic plots:
\begin{itemize}\itemsep1em
\item \href{https://github.com/leeper/Rcourse/tree/gh-pages/Scripts/summaryplots.r}{``Summary plots''}
\item \href{https://github.com/leeper/Rcourse/tree/gh-pages/Scripts/plotcolors.r}{``Plotting colors''}
\end{itemize}
}
% Maybe something here on graphical parameters

\againframe{q}



\section{Data Manipulation}
\frame{\tableofcontents[currentsection]}


\frame{
\frametitle{Recoding vectors}
\begin{itemize}\itemsep2em
\item Recoding is all about indexing
\item Several different ways to do it.
\item The \textbf{car} package has a nice function: \texttt{recode}:\\
\texttt{outvec <- recode(invec, "old1=new1; old2=new2; else=NA")}
\end{itemize}
}

\frame{\frametitle{Try on your own}
Understand recoding:
\begin{itemize}\itemsep1em
\item \href{https://github.com/leeper/Rcourse/tree/gh-pages/Scripts/recoding.r}{``Vector recoding''}
\end{itemize}
}

\againframe{q}


\frame{
\frametitle{Scale Construction}
\begin{itemize}\itemsep2em
\item Vectorization makes scaling easy
\item Use the usual operators: \texttt{+ - * / \^}
\item Convenience functions for sums and means
\end{itemize}
}

\frame{\frametitle{Try on your own}
Understand scale construction:
\begin{itemize}\itemsep1em
\item \href{https://github.com/leeper/Rcourse/tree/gh-pages/Scripts/basicscales.r}{``Basic Scale Construction''}
\end{itemize}
}

\againframe{q}



\frame{
\frametitle{Missing Data}
\begin{itemize}\itemsep2em
\item R has one missing data value: \texttt{NA}
\item Best to handle missing data during preprocessing
\end{itemize}
}

\frame{\frametitle{Try on your own}
Understand Missing Data:
\begin{itemize}\itemsep1em
\item \href{https://github.com/leeper/Rcourse/blob/gh-pages/Scripts/NA.r}{``Missing Data''}
\item \href{https://github.com/leeper/Rcourse/tree/gh-pages/Scripts/NAhandling.r}{``Handling Missing Data''}
\item \href{https://github.com/leeper/Rcourse/tree/gh-pages/Scripts/mi.r}{``Multiple Imputation''}
\end{itemize}
}

\againframe{q}


%\appendix

\bgroup
\setbeamercolor{background canvas}{bg=black}
\setbeamertemplate{navigation symbols}{}
\begin{frame}[plain]{}
\end{frame}
\egroup

\end{document}

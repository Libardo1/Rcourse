%\setbeameroption{notes on second screen} %Dual-Screen Notes
%\setbeameroption{show only notes} %Notes Output

% white on black
\setbeamercolor{normal text}{fg=white,bg=blue!20!black}
\setbeamercolor{structure}{fg=white}
\setbeamercolor{alerted text}{fg=red!85!black}
%\setbeamercolor{item projected}{use=item,fg=black,bg=item.fg!35}
\setbeamercolor*{palette primary}{use=structure,fg=structure.fg}
\setbeamercolor*{palette secondary}{use=structure,fg=structure.fg!95!black}
\setbeamercolor*{palette tertiary}{use=structure,fg=structure.fg!90!black}
\setbeamercolor*{palette quaternary}{use=structure,fg=structure.fg!95!black,bg=black!80}
\setbeamercolor*{framesubtitle}{fg=white}
\setbeamercolor*{block title}{parent=structure,bg=black!60}
\setbeamercolor*{block body}{fg=black,bg=black}
\setbeamercolor*{block title alerted}{parent=alerted text,bg=black!15}
\setbeamercolor*{block title example}{parent=example text,bg=black!15}

% black on white
%\usetheme{Singapore} %Gray with fade at top
%\usecolortheme{seagull} %Color theme
%\usecolortheme{rose} %Inner color theme
\setbeamercolor{item}{fg=black!65}
%\setbeamercolor{enumerate item}{fg=black!55}

\useoutertheme[subsection=false]{miniframes} %Supppress subsection in header
\useinnertheme{rectangles} %Itemize/Enumerate boxes
\setbeamertemplate{navigation symbols}{}
\setbeamertemplate{mini frames}[default]
\setbeamercovered{dynamics}
\setbeamerfont*{title}{size=\Large,series=\bfseries}
%\setbeamertemplate{headline}{}

\newcommand{\heading}[1]{\noindent \textbf{#1}\\ \vspace{1em}}

\usepackage{bbding,color,multirow,times,ccaption,tabularx,graphicx,verbatim} %graphics,
\usepackage{colortbl}%Table overlays
\usepackage[english]{babel}

% font
\usepackage{tgadventor}
\renewcommand*\familydefault{\sfdefault} %% Only if the base font of the document is to be sans serif
\usepackage[T1]{fontenc}


\title[]{Regression}

\begin{document}

\begin{frame}
	\titlepage
\end{frame}

\frame{}
\frame{\tableofcontents}

\section{Models}
\frame{\tableofcontents[currentsection]}

\frame{
\frametitle{Model Formul\ae}
\begin{itemize}\itemsep2em
\item Express relationships as a formula
\item Typically:\\
\texttt{y ~ x1 + x2}
\item Interactions:\\
\texttt{y ~ x1 * x2}\\
\texttt{y ~ x1 + x2 + x1:x2}
\item Formula is an object
\end{itemize}
}

\frame{\frametitle{Try on your own}
Understand formulae:
\begin{itemize}\itemsep1em
\item \href{https://github.com/leeper/Rcourse/tree/gh-pages/Scripts/formulae.r}{``Model Formulae''}
\end{itemize}
}

\frame[label=q]{\Large Questions so far?}

\frame{
\frametitle{Running a regression}
\begin{itemize}\itemsep2em
\item We already ran regressions earlier
\item Let's understand what we did
\end{itemize}
}

\frame[label=cis]{
\frametitle{The regression from earlier}
\noindent\texttt{
lm3\_1 <- lm(thresh\textasciitilde threat + fragdum,\\ data = cis)\\
\vspace{1em}
lm3\_2 <- lm(thresh\textasciitilde threat + fragdum,\\ data=cis[cis\$oursmpl==1,])\\
\vspace{1em}
lm3\_3 <- lm(thresh\textasciitilde threat13 + fragdum,\\ data=cis[cis\$oursmpl==1,])\\
\vspace{1em}
lm3\_4 <- lm(thresh\textasciitilde stthroct2 + fragdum,\\ data=cis[cis\$oursmpl==1,])\\
}
}

\frame{
\frametitle{Running a regression}
\begin{itemize}\itemsep2em
\item The linear regression function is \texttt{lm}
\item Specify a \texttt{formula} and a place to look for \texttt{data}
\item The result is an object of class ``lm''
\end{itemize}
}

\againframe{cis}


\section{Objects and Output}
\frame{\tableofcontents[currentsection]}

\frame{
\frametitle{Printing}
\begin{itemize}\itemsep2em
\item<1-> When we run a regression, we get some default output:\\
\texttt{lm(thresh\textasciitilde threat + fragdum,\\ data = cis)\\}

\item<2-> This doesn't print if we store the results:\\
\texttt{lm3\_1 <- lm(thresh\textasciitilde threat + fragdum,\\ data = cis)}

\item<3-> But we can see the results again, once stored, by simply calling:\\
\texttt{lm3\_1}
\end{itemize}
}



\frame{
\frametitle{Structure and summary}
\begin{itemize}\itemsep2em
\item Structure: \texttt{str(lm3\_1)}
\item Names: \texttt{names(lm3\_1)}
\item Summary: \texttt{summary(lm3\_1)}
\end{itemize}
}


\frame{
\frametitle{Coefficients}
\begin{itemize}
\item Coefficients:\\
\texttt{coef(lm3\_1)}\\
\texttt{lm3\_1\$coef}\\
\texttt{coef(summary(lm3\_1))}\\
\texttt{summary(lm3\_1)\$coef}
\end{itemize}
}

\frame{\frametitle{Try on your own}
Understand OLS coefficient plotting:
\begin{itemize}\itemsep1em
\item \href{https://github.com/leeper/Rcourse/tree/gh-pages/Scripts/olscoefplot.r}{``OLS Coefficients''}
\end{itemize}
}


\frame[label=q]{\Large Questions so far?}


\frame{
\frametitle{Writing Output to Word}
\begin{itemize}\itemsep2em
\item Writing a CSV
\item Using the \textbf{rtf} package:\\
\texttt{install.packages(\symbol{34}rtf\symbol{34})}\\
\texttt{library(rtf)}
\end{itemize}
}

\frame{\frametitle{Try on your own}
Output results to Word:
\begin{itemize}\itemsep1em
\item \href{https://github.com/leeper/Rcourse/tree/gh-pages/Scripts/wordoutput.r}{``Word output''}
\end{itemize}
}

\frame[label=q]{\Large Questions so far?}


\section{Interactions}
\frame{\tableofcontents[currentsection]}

\frame{
\frametitle{Let's work together}
\begin{itemize}\itemsep1em
\item \href{https://github.com/leeper/Rcourse/tree/gh-pages/Scripts/olsinteractionplots.r}{``Binary interactions''}
\item \href{https://github.com/leeper/Rcourse/tree/gh-pages/Scripts/olsinteractionplots2.r}{``Continuous interactions''}
\end{itemize}
}


%\appendix

\bgroup
\setbeamercolor{background canvas}{bg=black}
\setbeamertemplate{navigation symbols}{}
\begin{frame}[plain]{}
\end{frame}
\egroup

\end{document}
